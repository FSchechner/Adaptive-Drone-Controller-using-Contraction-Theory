\documentclass[conference]{IEEEtran}
\usepackage{amsmath,amssymb,amsfonts}
\usepackage{graphicx}
\usepackage{bm}
\usepackage{algorithm}
\usepackage{algorithmic}

\title{Comparative Analysis of PD and Lyapunov-Based Adaptive Control for Quadrotor Position Tracking Under Uncertainty}

\author{\IEEEauthorblockN{Florian Schechner}
\textit{Massachusetts Institute of Technology} \\
Course 2.165: Robotics \\
Cambridge, MA, USA \\
fschechner@mit.edu}

\begin{document}
\maketitle

\begin{abstract}
This paper presents a comparative analysis of classical PD control versus Lyapunov-based adaptive control for quadrotor position tracking under parametric uncertainty. A simplified 6-DOF translational dynamics model is employed, assuming direct force generation. The adaptive controller uses a regressor-based formulation to estimate the inverse mass ($\alpha = 1/m$) and three-dimensional disturbance vector online. A rigorous Lyapunov stability proof demonstrates global boundedness of parameter estimates and asymptotic tracking convergence. Batch simulations on a 20-second 3D slalom trajectory compare both controllers across mass and wind uncertainty. The adaptive controller maintains low tracking error from baseline (0.046 m) to worst-case heavy+wind (0.144 m), while PD degrades from 0.120 m to 0.387 m, demonstrating the value of online adaptation. However, disturbance adaptation remains imperfect—likely due to limited excitation and coupling between mass and specific-disturbance estimation—leading to residual bias under strong winds.
\end{abstract}

\begin{IEEEkeywords}
Adaptive Control, Lyapunov Stability, Quadrotor Control, PD Control, Parameter Estimation, Disturbance Rejection
\end{IEEEkeywords}

\section{Introduction}

Quadrotor unmanned aerial vehicles (UAVs) are increasingly deployed for autonomous delivery, inspection, and surveillance tasks. These applications often involve significant variations in operating conditions: payload attachment/release can change the vehicle mass by 30-50\%, and environmental disturbances from wind can introduce persistent forces of several Newtons. Classical fixed-gain controllers, while simple to implement, exhibit degraded performance under such parametric uncertainties.

Adaptive control offers a principled alternative by estimating unknown parameters online. For quadrotors, the translational dynamics can be written in a linearly parameterized form where the unknowns are the inverse mass and additive disturbances. This structure enables application of classical Lyapunov-based adaptive control techniques with guaranteed stability properties.

This work makes the following contributions:
\begin{enumerate}
\item A simplified 6-DOF position control formulation assuming direct force control, enabling focused comparison of position control strategies.
\item A regressor-based adaptive controller with rigorous Lyapunov proof of stability under mass and disturbance uncertainty.
\item Comprehensive simulation comparison between optimized PD and adaptive control on aggressive trajectories (20 s slalom) and a 16-run batch over mass/wind combinations.
\item Empirical analysis showing AC outperforms PD in aggregate, but disturbance adaptation remains imperfect under strong winds, likely due to excitation limits and coupling of mass and specific-disturbance estimation.
\end{enumerate}

\section{Dynamic Model}

\subsection{Simplified Translational Dynamics}

We consider a simplified quadrotor model where the vehicle can generate control forces directly in the inertial frame. The state vector is:
\[
\mathbf{x} = [\mathbf{p}^\top, \; \dot{\mathbf{p}}^\top]^\top = [x, y, z, \; v_x, v_y, v_z]^\top \in \mathbb{R}^6,
\]
where $\mathbf{p} = [x,y,z]^\top$ is the position in the inertial frame and $\dot{\mathbf{p}} = \mathbf{v}$ is the velocity.

The control input is a three-dimensional force vector:
\[
\mathbf{F}_{\text{control}} = [F_x, F_y, F_z]^\top \in \mathbb{R}^3.
\]

\subsection{Equations of Motion}

The translational dynamics follow Newton's second law:
\begin{equation}
m\ddot{\mathbf{p}} = \mathbf{F}_{\text{control}} - mg\mathbf{e}_3 + \mathbf{d},
\label{eq:newton}
\end{equation}
where:
\begin{itemize}
\item $m$ is the vehicle mass (unknown or varying)
\item $g = 9.81$ m/s$^2$ is gravitational acceleration
\item $\mathbf{e}_3 = [0, 0, 1]^\top$ is the vertical unit vector
\item $\mathbf{d} = [d_x, d_y, d_z]^\top$ represents unknown external disturbances (e.g., wind)
\end{itemize}

\subsection{Parameterized Form}

Define the \textit{inverse mass parameter}:
\begin{equation}
\alpha \triangleq \frac{1}{m},
\label{eq:alpha}
\end{equation}
and the \textit{specific disturbance} (disturbance per unit mass):
\begin{equation}
\hat{\mathbf{d}} \triangleq \frac{1}{m}\mathbf{d} = [\hat{d}_x, \hat{d}_y, \hat{d}_z]^\top.
\label{eq:dhat}
\end{equation}

Then equation (\ref{eq:newton}) can be rewritten as:
\begin{equation}
\ddot{\mathbf{p}} = \alpha \mathbf{F}_{\text{control}} - g\mathbf{e}_3 + \hat{\mathbf{d}}.
\label{eq:param_dynamics}
\end{equation}

This form is linear in the unknown parameters $\bm{\theta} = [\alpha, \hat{d}_x, \hat{d}_y, \hat{d}_z]^\top \in \mathbb{R}^4$, which enables Lyapunov-based adaptive control.

\section{Control Strategies}

\subsection{PD Position Controller}

The classical PD controller computes the desired acceleration as:
\begin{equation}
\mathbf{a}_{\text{des}} = -K_p \mathbf{e}_p - K_d \mathbf{e}_v,
\label{eq:pd_accel}
\end{equation}
where:
\begin{align}
\mathbf{e}_p &= \mathbf{p} - \mathbf{p}_d, \quad \text{(position error)} \\
\mathbf{e}_v &= \dot{\mathbf{p}} - \dot{\mathbf{p}}_d, \quad \text{(velocity error)}
\end{align}
and $K_p, K_d \in \mathbb{R}^{3\times 3}$ are diagonal gain matrices.

The velocity reference $\dot{\mathbf{p}}_d$ is computed via finite differences:
\begin{equation}
\dot{\mathbf{p}}_d(t) = \frac{\mathbf{p}_d(t) - \mathbf{p}_d(t-\Delta t)}{\Delta t}.
\end{equation}

The control force is:
\begin{equation}
\mathbf{F}_{\text{control}} = m_{\text{nom}}(\mathbf{a}_{\text{des}} + g\mathbf{e}_3),
\label{eq:pid_force}
\end{equation}
where $m_{\text{nom}}$ is the nominal (assumed) mass. Force saturation is applied:
\begin{equation}
\mathbf{F}_{\text{control}} = \text{clip}(\mathbf{F}_{\text{control}}, -F_{\max}, F_{\max}).
\end{equation}


\subsection{Lyapunov-Based Adaptive Controller}

\subsubsection{Sliding Surface Design}

Define the position and velocity tracking errors:
\begin{align}
\mathbf{e}_p &= \mathbf{p} - \mathbf{p}_d, \\
\mathbf{e}_v &= \dot{\mathbf{p}} - \dot{\mathbf{p}}_d.
\end{align}

Construct the sliding variable:
\begin{equation}
\mathbf{s} = \mathbf{e}_v + \Lambda \mathbf{e}_p,
\label{eq:sliding}
\end{equation}
where $\Lambda = \text{diag}(\lambda_x, \lambda_y, \lambda_z) > 0$ is a positive definite matrix. The sliding variable $\mathbf{s}$ represents a linear combination of position and velocity errors.

\subsubsection{Control Law}

The commanded acceleration is:
\begin{equation}
\mathbf{a}_{\text{cmd}} = -\Lambda \mathbf{e}_v - K \mathbf{s} + \ddot{p_d},
\label{eq:acmd}
\end{equation}
where $K = \text{diag}(k_x, k_y, k_z) > 0$ determines the convergence rate.

The desired total acceleration (including gravity compensation) is:
\begin{equation}
\mathbf{a}_{\text{des}} = \mathbf{a}_{\text{cmd}} + g\mathbf{e}_3.
\label{eq:ades}
\end{equation}

Using the estimated parameters $\hat{\alpha}$ and $\hat{\mathbf{d}}$, the control force is:
\begin{equation}
\mathbf{F}_{\text{control}} = \hat{m} \mathbf{a}_{\text{des}} - \hat{m}  \hat{\mathbf{d}},
\label{eq:adaptive_force}
\end{equation}
where $\hat{m} = 1/\max(\hat{\alpha}, \epsilon)$ with $\epsilon = 10^{-6}$ to avoid division by zero.

\subsubsection{Regressor Formulation}

From \eqref{eq:param_dynamics},
\begin{equation}
\ddot{\mathbf{p}} = \alpha \mathbf{F}_{\text{control}} - g\mathbf{e}_3 + \hat{\mathbf{d}},
\end{equation}
adding the known gravity term to the left-hand side gives the linearly parameterized form
\begin{equation}
\ddot{\mathbf{p}} + g\mathbf{e}_3 = \alpha \mathbf{F}_{\text{control}} + \hat{\mathbf{d}}.
\label{eq:param_dynamics_reg}
\end{equation}

This can be written in regressor form:
\begin{equation}
\ddot{\mathbf{p}} + g\mathbf{e}_3 = Y(\mathbf{F}_{\text{control}})\,\bm{\theta},
\label{eq:regressor}
\end{equation}
where the regressor matrix is
\begin{equation}
Y(\mathbf{F}_{\text{control}}) =
\begin{bmatrix}
F_x & 1 & 0 & 0 \\
F_y & 0 & 1 & 0 \\
F_z & 0 & 0 & 1
\end{bmatrix} \in \mathbb{R}^{3 \times 4},
\label{eq:Y}
\end{equation}
and the parameter vector is
\begin{equation}
\bm{\theta} = [\alpha, \; \hat{d}_x, \; \hat{d}_y, \; \hat{d}_z]^\top.
\end{equation}

\subsubsection{Adaptive Law}

The parameter estimates are updated according to the standard gradient adaptive law
\begin{equation}
\dot{\hat{\bm{\theta}}} = \Gamma Y^\top(\mathbf{F}_{\text{control}})\,\mathbf{s},
\label{eq:adaptive_law}
\end{equation}
where $\Gamma = \text{diag}(\gamma_\alpha, \gamma_d, \gamma_d, \gamma_d) > 0$ is the adaptation gain matrix. 

Parameter projection enforces physical bounds:
\begin{align}
\hat{\alpha} &\in [\alpha_{\min}, \alpha_{\max}], \\
\hat{d}_i &\in [-d_{\max}, d_{\max}], \quad i \in \{x,y,z\}.
\end{align}


\section{Lyapunov Stability Analysis}

\subsection{Error Dynamics}

Taking the time derivative of the sliding variable (\ref{eq:sliding}):
\begin{align}
\dot{\mathbf{s}}
&= \dot{\mathbf{e}}_v + \Lambda \dot{\mathbf{e}}_p \nonumber \\
&= \ddot{\mathbf{p}} - \ddot{\mathbf{p}}_d + \Lambda \mathbf{e}_v.
\label{eq:sdot_step1}
\end{align}

Substituting the actual dynamics (\ref{eq:param_dynamics}),
\begin{equation}
\dot{\mathbf{s}}
= \alpha \mathbf{F}_{\text{control}}
- g\mathbf{e}_3
+ \hat{\mathbf{d}}
- \ddot{\mathbf{p}}_d
+ \Lambda \mathbf{e}_v.
\label{eq:sdot_step2}
\end{equation}

Using the control law (\ref{eq:acmd})--(\ref{eq:adaptive_force}), all known terms cancel in closed-loop, and the error dynamics take the standard adaptive form
\begin{equation}
\dot{\mathbf{s}}
= -K\mathbf{s}
+ Y(\mathbf{F}_{\text{control}})\,\tilde{\bm{\theta}},
\label{eq:error_dynamics}
\end{equation}
where $\tilde{\bm{\theta}} = \bm{\theta} - \hat{\bm{\theta}}$ is the parameter estimation error.
\subsection{Lyapunov Function}

Consider the Lyapunov function
\begin{equation}
V = \tfrac{1}{2}\mathbf{s}^\top\mathbf{s}
  + \tfrac{1}{2}\tilde{\bm{\theta}}^\top\Gamma^{-1}\tilde{\bm{\theta}} .
\end{equation}
Using (\ref{eq:error_dynamics}) and (\ref{eq:adaptive_law}),
\[
\dot{V}
= -\mathbf{s}^\top K \mathbf{s} \le 0,
\]
which proves stability and convergence of $\mathbf{s}(t)$.

\subsection{Time Derivative and Adaptive Law Design}

Taking the time derivative of the Lyapunov function:
\begin{align}
\dot{V} &= \mathbf{s}^\top \dot{\mathbf{s}} + \tilde{\bm{\theta}}^\top \Gamma^{-1} \dot{\tilde{\bm{\theta}}} \nonumber \\
&= \mathbf{s}^\top \left(-K\mathbf{s} + Y \tilde{\bm{\theta}}\right) - \tilde{\bm{\theta}}^\top \Gamma^{-1} \dot{\hat{\bm{\theta}}} \nonumber \\
&= -\mathbf{s}^\top K \mathbf{s} + \underbrace{\tilde{\bm{\theta}}^\top Y^\top \mathbf{s}}_{\text{cross term 1}} - \underbrace{\tilde{\bm{\theta}}^\top \Gamma^{-1} \dot{\hat{\bm{\theta}}}}_{\text{cross term 2}},
\label{eq:vdot_step1}
\end{align}
where we used $\dot{\tilde{\bm{\theta}}} = -\dot{\hat{\bm{\theta}}}$ since $\bm{\theta}$ is constant, and the symmetry $\mathbf{s}^\top Y \tilde{\bm{\theta}} = \tilde{\bm{\theta}}^\top Y^\top \mathbf{s}$.

The goal is to \textit{choose} the parameter update law $\dot{\hat{\bm{\theta}}}$ such that the two cross terms cancel each other, leaving only the negative definite term $-\mathbf{s}^\top K \mathbf{s}$. Setting the cross terms equal:
\begin{equation}
\tilde{\bm{\theta}}^\top Y^\top \mathbf{s} = \tilde{\bm{\theta}}^\top \Gamma^{-1} \dot{\hat{\bm{\theta}}}.
\end{equation}

This is satisfied by choosing:
\begin{equation}
\boxed{\dot{\hat{\bm{\theta}}} = \Gamma Y^\top \mathbf{s}}.
\label{eq:adaptive_choice}
\end{equation}

Substituting (\ref{eq:adaptive_choice}) into (\ref{eq:vdot_step1}), the two cross terms cancel exactly:
\begin{align}
\dot{V} &= -\mathbf{s}^\top K \mathbf{s} + \tilde{\bm{\theta}}^\top Y^\top \mathbf{s} - \tilde{\bm{\theta}}^\top \Gamma^{-1} \dot{\hat{\bm{\theta}}} \nonumber \\
&= -\mathbf{s}^\top K \mathbf{s} + \tilde{\bm{\theta}}^\top Y^\top \mathbf{s} - \tilde{\bm{\theta}}^\top \Gamma^{-1} \left(\Gamma Y^\top \mathbf{s}\right) \nonumber \\
&= -\mathbf{s}^\top K \mathbf{s} + \tilde{\bm{\theta}}^\top Y^\top \mathbf{s} - \tilde{\bm{\theta}}^\top Y^\top \mathbf{s} \nonumber \\
&= -\mathbf{s}^\top K \mathbf{s}.
\label{eq:vdot_final}
\end{align}

Therefore, $\dot{V} = -\mathbf{s}^\top K \mathbf{s} \leq 0$, establishing that the Lyapunov function is strictly non-increasing along all system trajectories. The cross terms cancel exactly due to the specific choice of the adaptive law.

\subsection{Stability Conclusions}

\textbf{Theorem 1 (Stability):} Under the adaptive control law (\ref{eq:adaptive_force}) with parameter update (\ref{eq:adaptive_choice}):
\begin{enumerate}
\item The tracking errors converge asymptotically: $\mathbf{e}_p(t), \mathbf{e}_v(t) \to 0$ as $t \to \infty$.
\item Parameter estimates remain bounded but may not converge to true values without persistent excitation.
\end{enumerate}

\textit{Proof:}
Since $V(\mathbf{s}, \tilde{\bm{\theta}}) > 0$ and $\dot{V} = -\mathbf{s}^\top K \mathbf{s} \leq 0$, we have $V(t) \leq V(0)$ for all $t \geq 0$. This implies both $\mathbf{s}$ and $\tilde{\bm{\theta}}$ are bounded. Integrating from 0 to $\infty$:
\[
\int_0^\infty \mathbf{s}^\top K \mathbf{s} \, dt \leq V(0) < \infty,
\]
which shows $\mathbf{s} \in \mathcal{L}_2 \cap \mathcal{L}_\infty$. By Barbalat's lemma, since $\dot{\mathbf{s}}$ is bounded, we conclude $\mathbf{s}(t) \to 0$ as $t \to \infty$. From the sliding variable definition $\mathbf{s} = \mathbf{e}_v + \Lambda \mathbf{e}_p$ and $\mathbf{s} \to 0$, the tracking errors satisfy $\dot{\mathbf{e}}_p + \Lambda \mathbf{e}_p \to 0$. Since $\Lambda > 0$, this implies exponential convergence: $\mathbf{e}_p, \mathbf{e}_v \to 0$ as $t \to \infty$. Parameter estimates $\hat{\bm{\theta}}$ remain bounded by the Lyapunov argument but may not converge to true values without persistent excitation. \qed

\section{Implementation Details}

\subsection{Gain Optimization}

Both controllers were optimized using Nelder-Mead simplex optimization on a 10-second slalom trajectory. The optimization framework was implemented using Claude Code, an AI-assisted development tool. The objective function minimized mean tracking error:
\begin{equation}
J = \frac{1}{N} \sum_{i=1}^N \|\mathbf{p}(t_i) - \mathbf{p}_d(t_i)\|.
\end{equation}

\textbf{Optimized PD Gains:}
\begin{align}
K_{p,xy} &= 60.00, \quad K_{p,z} = 60.00, \\
K_{d,xy} &= 3.82, \quad K_{d,z} = 8.84, \\
K_{i,xy} &= 0.50, \quad K_{i,z} = 0.50.
\end{align}

\textbf{Optimized Adaptive Gains:}
\begin{align}
\lambda_{xy} &= 11.99, \quad \lambda_z = 14.36, \\
k_{xy} &= 24.93, \quad k_z = 21.74, \\
\gamma_\alpha &= 0.34, \quad \gamma_d = 0.99.
\end{align}

\subsection{Trajectory Design}

The evaluation uses a 20-second 3D slalom trajectory with variable forward speed. The reference is:
\begin{itemize}
\item $x(t) = A_x \sin(\omega_x t)$ with $A_x = 5$ m, $\omega_x = 1.2$ rad/s.
\item $y(t)$ uses a time-varying forward speed $v_y(t) = 1.5 + 0.8\sin(0.5 t)$.
\item $z(t) = z_0 + A_z \sin(\omega_z t)$ with $A_z = 1$ m, $\omega_z = 0.8$ rad/s.
\end{itemize}
This trajectory excites all axes and introduces variation in forward speed to test tracking under persistent motion.

\subsection{Test Scenarios}

Batch evaluation comprises 16 runs:
\begin{enumerate}
\item \textbf{Baseline:} Nominal mass ($m = 1.9$ kg), no disturbance.
\item \textbf{Mass sweep:} Five masses from 1.9 to 4.0 kg, no wind.
\item \textbf{Wind sweep:} Five winds along direction $[-1, 2, 0]$ scaled from 0 to 10 N at nominal mass.
\item \textbf{Combined:} Matching mass and wind levels from the above sets.
\end{enumerate}
Performance is reported as mean total tracking error per scenario and averaged across the batch.

\section{Simulation Results}

\subsection{Quantitative Performance}

Table \ref{tab:results} summarizes the mean absolute tracking errors across all test scenarios.

\begin{table}[h]
\centering
\caption{Mean Absolute Tracking Errors [m] (slalom batch)}
\label{tab:results}
\begin{tabular}{|l|c|c|}
\hline
\textbf{Scenario} & AC $e_{\text{total}}$ & PD $e_{\text{total}}$ \\
\hline
Baseline (1.9 kg, no wind) & 0.046 & 0.120 \\
Heavy + wind (4.0 kg, 10 N) & \textbf{0.144} & \textbf{0.387} \\
\hline
\end{tabular}
\end{table}

\subsection{Performance Analysis}

\begin{enumerate}
\item \textbf{Baseline Performance:} In the nominal scenario without disturbances, AC achieves 62\% lower tracking error (0.046 m vs 0.120 m) than PD.

\item \textbf{Mass/Wind Robustness:} With mass increased to 4.0 kg and wind up to 10 N along direction $[-2, 4, 0]$, AC error remains at 0.144 m while PD grows to 0.387 m. Disturbance adaptation is incomplete, indicating excitation limits and coupling between mass and specific-disturbance estimation.

\item \textbf{Cumulative Robustness:} Across the batch of 16 scenarios (mass, wind, combined), AC maintains a lower average total error than PD, demonstrating the benefit of online parameter adaptation under simultaneous uncertainties.
\end{enumerate}

\section{Discussion}

\subsection{Advantages of Adaptive Control}

\begin{enumerate}
\item \textbf{Guaranteed Stability:} Lyapunov analysis provides formal stability guarantees regardless of parameter uncertainty within bounded regions.

\item \textbf{No Manual Retuning:} A single set of gains works across wide mass range (1.9-2.9 kg), eliminating gain scheduling.

\item \textbf{Disturbance Adaptation:} Online estimation of $\hat{\mathbf{d}}$ enables rejection of persistent disturbances, though performance degrades under strong winds when excitation is limited.

\item \textbf{Predictable Transients:} Convergence rate is determined by $\Lambda$ and $K$, enabling systematic performance tuning.
\end{enumerate}

\subsection{Limitations and Future Work}

\begin{enumerate}
\item \textbf{Persistent Excitation:} Parameter convergence to true values requires trajectory richness. Lack of excitation may result in biased estimates; richer excitation and decoupling mass and disturbance adaptation could improve strong-wind performance.

\item \textbf{Computational Cost:} Adaptive law requires regressor computation and matrix operations at each timestep, slightly increasing computational burden vs. PD.

\item \textbf{Experimental Validation:} Simulation results should be validated on physical hardware with real sensors, actuators, and environmental disturbances.
\end{enumerate}

\section{Conclusion}
This paper presented a comparative analysis of PD and Lyapunov-based adaptive control for quadrotor position tracking under parametric uncertainty. A simplified 6-DOF translational model enabled focused evaluation of position control strategies. Batch simulations on a 20-second slalom with mass and wind sweeps showed AC outperforming PD (baseline 0.046 m vs 0.120 m; heavy+wind 0.144 m vs 0.387 m), but disturbance adaptation remained imperfect under strong winds, likely due to limited excitation and coupling with mass adaptation. The adaptive approach eliminates gain scheduling and manual retuning while maintaining provable stability guarantees, with room to improve excitation and decouple mass/disturbance estimation. 

\section{Code Availability}

All simulation code, controller implementations, and optimization scripts are publicly available on GitHub: \texttt{https://github.com/FSchechner/Adaptive-Drone-Controller-using-Lyapunov-Theory}

To reproduce the results presented in this paper:
\begin{enumerate}
\item Clone the repository
\item Navigate to the \texttt{simulator} directory
\item Run: \texttt{python3 simple.py}
\end{enumerate}

This will execute all three test scenarios and generate the comparison plot.

\bibliographystyle{IEEEtran}
\begin{thebibliography}{1}

\bibitem{slotine1991}
J.-J. E. Slotine and W. Li, \textit{Applied Nonlinear Control}. Prentice Hall, 1991.

\end{thebibliography}

\end{document}
